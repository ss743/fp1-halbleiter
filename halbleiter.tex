\documentclass[12pt]{article}

\usepackage{fancyhdr}
\usepackage{geometry}
\usepackage{ucs}
\usepackage[utf8x]{inputenc}
\usepackage[T1]{fontenc}
\usepackage[ngerman]{babel}
\usepackage{amsmath,amssymb,amstext}
\usepackage{hyperref}
\usepackage{cancel}
\usepackage{dsfont}
\usepackage{physics}
\usepackage{lmodern}
\usepackage{enumerate}
\usepackage{enumitem}
\usepackage{graphicx}
\usepackage{listings, color}
\usepackage[labelfont=bf]{caption}
\usepackage{titling}



\lstset{basicstyle=\scriptsize,breaklines=true} %Quellcode mit Umlauten und ganz klein

\lstset{literate=
  {Ö}{{\"O}}1
  {Ä}{{\"A}}1
  {Ü}{{\"U}}1
  {ß}{{\ss}}2
  {ü}{{\"u}}1
  {ä}{{\"a}}1
  {ö}{{\"o}}1
}


%Geometrie----------------------------------------------------------------------------------------------------------

\geometry{a4paper, top=25mm, left=15mm, right=15mm, bottom=25mm,headsep=10mm, footskip=10mm}
\pagestyle{fancy}
\setlength{\parindent}{0pt} %Zeileneinrückung

\fancyhf{} %Setzt voreingestellte Kopf-und Fußzeilen-Eigenschaften zurück

\lhead{\nouppercase{\leftmark}}
\chead{}
\rhead{\thepage}

\lfoot{}
\cfoot{}
\rfoot{}

\title{\vspace{0cm}{\Huge Fortgeschrittenen-Praktikum I:\\ \vspace{1cm}  Halbleiter}}
\author{Saskia Bondza\\Simon Stephan}
\date{Durchgeführt am 07.09.2016}

\pretitle{%
  \begin{center}
  \LARGE
  \includegraphics[width=6cm,]{figures/siegel}\\[\bigskipamount]
}
\posttitle{\end{center}}

%neue Commands----------------------------------------------------------------------------------------------------------
\newcommand{\nab}{\vec{\nabla}} %direkter Befehl mit Vektorpfeil
\newcommand{\gra}[3][0.7]{
	\begin{minipage}[h!]{\textwidth}
		\centering
		\includegraphics[width=#1\textwidth]{figures/#2.png}
		\captionof{figure}{#3}
	\end{minipage}
	\vskip 30 pt
	}
\newcommand{\graTwo}[4][0.5]{
	\begin{minipage}[h!]{\textwidth}
		\centering
		\includegraphics[width=#1\textwidth]{figures/#2.png}
		\includegraphics[width=#1\textwidth]{figures/#3.png}
		\captionof{figure}{#4}
	\end{minipage}
	\vskip 30 pt
}
\newcommand{\code}[1]{\texttt{#1}}

%Titel,Inhalt----------------------------------------------------------------------------------------------------------

\begin{document}
\pagenumbering{gobble} %verstecke Seitenzahl
\maketitle
\newpage

\thispagestyle{empty}
\tableofcontents
\newpage

%Schreiben----------------------------------------------------------------------------------------------------------
\pagenumbering{arabic} %verstecke Seitenzahl
\section{Einleitung}

In diesem Experiment soll in drei verschiedenen Versuchsteilen die Funktionsweise von Halbleitern und Halbleiterdetektoren untersucht werden. Silizium wird hier als Beispiel für einen reinen Halbleiter untersucht und Kadmiumtellurid als Beispiel für einen Verbindungshalbleiter.
In \textbf{Teil I} des Versuchs wird die Bandlückenenergie von Silizium und Germanium, wichtiges Charakteristikum von Halbleitern, gemessen, die die Auflösung eines Halbleiterdetektors bestimmt.\\
In \textbf{Teil II} des Versuchs geht es um die Beobachtung der Ausbreitung
von Ladungsträgern im Halbleiter mit dem Haynes-Shockley-Experiment und in \textbf{Teil III} des Experiments werden mit Halbleiterdetektoren radioaktive Spektren aufgenommen und untersucht, wie gut sich Halbleiterdetektoren hierfür eignen.


\newpage
\section[Theoretische Grundlagen]{Theoretische Grundlagen}

Dieser Teil orientiert sich an \cite{staat}.



\subsection{Interferenz von Photonen, optisches Gitter}

Trifft eine ebene Welle auf einen Spalt, so ist nach dem Huygene'schen Prinzip jeder Punkt der Wellenfront Ausgangspunkt einer Elementarwelle, wobei sich die neue Wellenfront durch Überlagerung sämtlicher Elementarwellen ergibt. Trifft eine ebene Welle auf mehrere Spalte, ein optisches Gitter, beobachtet man hinter dem Gitter Interferenz, ein charakteristisches Wellenphänomen.  Ist der Gangunterschied zweier Wellen ein ganzzahliges Vielfaches ihrer Wellenlänge kommt es zur konstruktiven Interferenz, ist der Gangunterschied ein ungerades, halbzahliges Vielfaches der Wellenlänge kommt es zur destruktiven Interferenz. Es gilt folgender Zusammenhang für die konstruktive Interferenz am Gitter:

\begin{align*}
& n\cdot\lambda=d\cdot\sin(\theta)
\end{align*}
Hierbei ist $n$ die Beugungsordnung, $\theta$ der Einfallswinkel, $\lambda$ die Wellenlänge und $d$ der Spaltabstand.
Man unterscheidet zwei Arten von optischen Gittern: Transmissionsgitter und Reflexionsgitter; Transmissionsgitter bestehen aus parallelen Spalten, Reflexionsgitter aus reflektierenden Graten. Hier handelt es sich um ein Reflexionsgiterr. Wir betrachten nun nur die erste Beugungsordnung. Für verschiedene Anstellwinkel des optischen Gitters ergibt sich die Energien der Photonen zu:

\[E(\varphi)=\frac{hc}{2d\sin(\varphi)cos(\frac{\psi}{2})}\]

In \ref{4.1} wird genauer erklärt wie sich ein optisches Gitter so als Spektrometer verwenden lässt.

\subsection{Halbleiter}

\subsection{Das Bändermodell}

Bei einem einzelnen Atome sind die Energieniveaus der Elektronen diskret. Nähert man nun zwei Atome so nah an, dass sich ihre Orbitale überlappen, müssen sich die Energieniveaus gemäß dem Pauliprinzip aufspalten. In einem Kristall wechseln nun der Art viele Atome miteinander, dass auf diese Art und Weise Energiebänder enstehen (s. Abbildung \ref{Bandlucke}). Das energetisch höchste Band, das im Grundzustand besetztist heißt Valenzband, das erste unbesetzte Band heißt Leitungsband. Der Energieabstand zwischen diesen beiden Bändern wird als Bandlücke bezeichnet. Bei $T = 0$K ist das Valenzband voll besetzt während im Leitungsband keine Ladungsträger sind. Durch Anlegen einer Spannung oder Absorption von Photonen können Elektronen jedoch in das Leiterband gelangen. 

\gra{Bandlucke}{Bändermodell des Halbleiters \label{Bandlucke}}
Die charakteristische Größe der Bandlückenenergie definiert ob ein Stoff ein Isolator, Halbleiter oder Leiter ist. Isolatoren haben eine besonders große Bandlüclenenergie, die mehrere $eV$ beträgt. Halbleiter haben typischerweise eine Bandlücke von weniger als einem $eV$, sie isolieren nur bei sehr niedrigen Temperaturen. Bereits thermische Anregungen reichen bei höheren Temperaturen für die Elektronen aus, um ins Leitungsband gehoben zu werden, Zurück bleibt bei diesem Vorgang ein positiv geladenes Loch, man spricht von einem Elektron-Loch-Paar. Bei Leitern überlappen sich beide Energiebänder, sodass stets genug Ladungsträger im Leitungsband sind, um einen Stromfluss zu gewährleisten.

\subsection{Bewegung von Ladungsträgern im Halbleiter}

\subsection{Direkte und indirekte Halbleiter}

Untersucht man die Energien der Bänder in Abhängikeit des Impulses ergeben sich komplizierte Verläufe, aus denen sich eine natürliche Unterscheidung von Halbleitern ableiten lässt: Liegt das Maximum des Valenzbandes beim gleichen Impuls wie das Minimum des Leitungsbandes spricht man von direkten Halbleitern, liegen Maximum und Minimum bei verschiedenen Impulsen, spricht man von indirekten Halbleitern.

\gra{(in)direkte_Halbleiter}{Direkte und indirekte Halbleiter}
Bei direkten Halbleitern können Elektronen einfach durch die Aufnahme der Bandlückenenergie (Übertrag durch ein Photon von ausreichender Energie $E_{Photon}=\hbar\omega\geq E_g$) ins Leitungsband gehoben werden während bei indirekten Halbleitern durch die Verschiebung der Extrema um $\Delta p$ ein Elektron zusätzlich zur Energieaufnahme auch noch seinen Impuls ändern muss. Dies geschieht durch Erzeugung bzw. Vernichtung von Phononen, d.h. Gitterschwingungen.

\subsection{Extrinsische Halbleiter und Dotierung}

In der Theorie wird meist von intrinsischen Halbleitern ausgegangen, also perfekten Kristallen ausgegangen.  Diese lassen sich allerdings technisch nicht erzeugen. Kristalldefekte, z.B.Beschädigte Elementarzellen, Verschiebung ganzer Kristallebenen oder die Verunreinigung durch Fremdatom beeinflussen die Kristallstruktur und können zur Bildung von Energieniveaus innerhalb der Bandlücken führen, was einen nicht zu vernachlässigenden Effekt auf die Elektronen-Loch-Rekombination hat, und somit die Leitfähigkeit beeinflusst.\\
Normalerweise geht ein Halbleiteratom Bindungen mit vier Nachbaratomen ein. Bei der Verunreinigung mit Fremdatomen z.B. Phosphor mit fünf Valenzelektronen oder Aluminium mit drei Valenzelektronen steht ein quasi-freier Ladungsträger zur Verfügung: Bei Phosphor ist dies das fünfte, schwachgebundene Elektron, bei Aluminium steht ein zusätzliches Loch als Ladungsträger zur Verfügung. Bei Fremdatome, die ein zusätzliches Elektron beisteuern, wie in diesem Beispiel Phosphor, spricht man von Donatoren. Bei solchen, die ein zusätzliches Loch beisteuern, wie hier Aluminium, spricht man von Akzeptoren.
Oft wird diese Verunreinigung absichtlich vorgenommen (typischerweise ca. 1 Fremdatom auf $10^6$Atome), da sie sich positiv auf die Leitfähigkeit des Halbleiters auswirken kann. Diesen Vorgang nennt man Dotierung. Donatoren-Halbleiter nennt man n-Typ, Akzeptor-Halbleiter heißen p-Typ.
\subsection{Halbleiter-Diode}

Bringt man eine Schicht n-Typ-Halbleiter und eine Schicht p-Typ-Halbleiter zusammen, spricht man von einer Diode. Die quasie-freien Ladungsträger der n-Typ-Schicht, d.h. die Elektronen, diffundieren in die freien Löcher der p-Typ-Schicht. Im Übergangsbereich löschen sich die freien Ladungsträger also durch diesen Vorgang aus. 
Nach diesem Diffusionsprozess bleiben in der n-Typ-Schicht positiv geladene Ionen und in der p-Typ-Schicht negativ geladene Ionen zurück (Verarmungszone), die sich nicht frei bewegen können, da sie in die Gitterstruktur des Kristalls fest eingebunden sind. Es ensteht so ein elektrisches Feld (Kontakt-Potential mit Spannung $U_{bi}$), wie in Abbildung 3 zu sehen ist. 

\gra{n-p-Diode}{Halbleiter-Diode}

Durch Anlegen einer äußeren Spannung bei der der positive Pol an die n-Schicht angeschlossen wird, und der negative Pol an die p-Schicht vergrößert sich die Verarmungszone und die Ladungsträger müssen ein größeres Spannungsgefälle überwinden: die Diode ist in Sperrrichtung gepolt. Bei Anlegung einer äußeren Spannung mit umgekehrter Polung können die überschüssigen Elektronen in der p-Schicht problemlos zum positiven Pol wandern, die Diode ist in Durchlassrichtung gepolt.
Dies bildet die Grundlage für Halbleiterdetektoren. Die Grundidee von Halbleiterdetektoren ist gerade eine Halbleiterdiode in einen Stromkreis einzubinden.
\subsection{Wechselwirkung von Strahlung mit Materie \cite{anleitungkhwz}}

Die Wechselwirkung von Strahlung mit Materie wird im Wesentlichen durch drei verschiedene Effekte beschrieben: \textbf{Photoeffekt}, \textbf{Comptoneffekt} und \textbf{Paarbildung}.
Der Effekt der Paarbildung tritt für Energien über $1,022$ MeV auf, die in diesen Versuchen nicht erreicht werden, sodass wir diesen Effekt hier nicht weier erläutern.

\paragraph{Photoeffekt} Man unterscheidet beim Photoeffekt zwischen innerem und äußerem Photoeffekt.
Beim Photoeffekt dringt ein Photon in das Atom ein und überträgt seine gesamte Energie an ein Elektron der inneren Schalen. Dabei wird Energie auf dieses Elektron übertragen. Ist die Energie des Photons größer als die Bindungsenergie des Elektrons wird es aus der Atomhülle befreit und erhält kinetische Energie. Die hier entstandene Lücke wird über Abstrahlung eines $\gamma$-Quants oder eines Auger-Elektrons wieder gefüllt. Man spricht vom äußeren Photoeffekt.
Beim inneren Photoeffekt treffen Photonen mit einer Energie  $\geq E_g$ auf ein Elektron, und heben dieses vom Valenzband in das Leiterband. Hier durch wird die Leitfäjigkeit des Halbleiters erhöht.
\paragraph{Comptoneffekt} Beim Compton-Effekt trifft ein einfallendes Photon auf ein freies oder nur leicht gebundenes Elektron und überträgt einen Teil seiner Energie bzw. seines Impulses auf dieses und verliert selber an Energie in dem die Wellenlänge erhöht wird. Die Stärke dieses Effektes hängt vom Emissionswinkel ab, gestreute Photonen können weiter mit anderen Elektronen wechselwirken, beispielsweise in Form des inneren Photoeffekts. Dieser Prozess findet meist bei Energien zwischen $200$ keV und $5$ MeV statt.

Trifft hochenergetische Strahlung, z.B. in Form eines Lasers, auf einen Halbleiter, ensteht durch die oben beschriebenen Effekte eine Ladungsträgerwolke.

In Materie fällt die Intensität $I_d$ der $\gamma$-Strahlung exponentiell mit der Strecke $d$ ab:
\[I_d=e^{\mu d}\]
Dabei ist $\mu$ der Absorptionskoeffizent des Materials und $I_0$ die Anfangsintensität.

\subsection{Theorie zum Haynes-Shockley Experiment}

Im zweiten Veruchsteil des Experiments wird ein Laserpuls auf eine Germaniumprobe geschossen, wodurch eine Ladungswolke ensteht.Um das Verhalten dieser Ladungsträgerwolke zu beschreiben stellen wir folgende Überlegungen an: Aus der Kontinuitätsgleichung $\vec{\nabla}\vec{j}+\dot{n}=0$ und zusätzlichen Rekombinationen (siehe \cite{staat}), ergibt sich:
\begin{align*}
&\frac{\partial n}{\partial t}=-\vec{\nabla}\vec{j_n}\frac{n-n_o}{\tau_n}=\vec{E}\mu_n\vec{\nabla}n+D_n\triangle n-\frac{n-n_o}{\tau_n}&\mbox{(analog für die Löcher)}
\end{align*}
wobei  $\tau_n$ die  Rekombinationszeit ist und $n_0,p_0$ sind die Grundkonzentrationen, für die gilt: $n(\vec{x})=n_0+\bar{n}(\vec{x})$

Diese Differentialgleichung ist gekoppelt. Wir vereinfachen sie daher durch Einfhren allgemeiner Größen zu einer entkoppelten DGL:
\[\frac{\partial c}{\partial t}=\mu\vec{E}\cdot\nab c+D\triangle c-\frac{c-c_0}{\tau}\]
wobei :
\begin{itemize}
	\item $c=$  ambipolare Ladungsträgerdichte.
	\item $c_0=$  Grundkonzentration von $c$.
	\item $D=$  Diffusionskonstante.
	\item $\mu=$ Beweglichkeit.
\end{itemize}
Man kann zeigen, dass folgende Funktion diese Differenzialgleichung löst:
\[c(x,t)=Ce^{-\frac{t}{\tau_n}}\frac{1}{\sqrt{4\pi D_n t}}e^{-\frac{(x-\mu_n E t)^2}{4D_n t}}\]
Bei genauerem Hinsehen erkennt man schnell, dass es sich im Wesentlichen um eine Gaußfunktion handelt. Es macht daher Sinn, für die Auswertung dieses Versuchsteils Fit-Funktionen folgender Form zu verwenden:
\[c(x,t)=A(t)\cdot\frac{1}{\sqrt{2\pi\sigma^2}}\cdot e^{-\frac{(x-x_c(t))^2}{2\sigma^2}}\]

mit:
\begin{itemize}
	\item $A(t)=Ce^{-\frac{t}{\tau_n}}$
	\item $x_c(t)=\mu_n E t$
	\item $\sigma(t)=\sqrt{2 D_n t}$
\end{itemize}

\subsection{Geräte}

 
\newpage
\section[Versuchsaufbau- und Durchführung]{Versuchsaufbau- und Durchführung} 


\subsection{Teil I}

\subsubsection{Aufbau} \label{4.1}
\gra{TeilI_Aufbau}{Versuchsaufbau Teil I: Messen der Bandlückenenergie}\label{Aufbau1}

Abbildung \ref{Aufbau1} zeigt den schematischen Versuchsaufbau des Versuchs. Der Versuchsaufbau besteht aus einer optischen Bank, die in einem 15 Grad Winkel zueinander stehen. Am Beginn des Strahlengangs steht eine Lichtquelle, die ein kontinuierliches Lichtspektrum (weißes Licht) emittiert, es handelt sich dabei um eine Lampe. Mittels einer Linse die im Abstand ihrer Brennweite zur Lampe steht, wird dieses Licht parallelisiert und trifft auf ein optisches Gitter, das am Ende des ersten Armes montiert ist. Das optische Gitter ist auf einer Halterung drehbar montiert, sodass der Einstellwinkel varriert werden kann. Je nach Einstellwinkel fällt als Folge von Interferenz Licht einer anderen Wellenlänge auf die Probe, das optische Gitter fungiert als Spektrometer. Es steht für jede Probe je ein geeignetes Gitter zur Verfügung. Auf dem zweiten Arm der Bank trifft das Licht zunächst auf eine maximal $2$ cm geöffnete Blende. Mit einem optischen Filter soll UV-Strahlung zweiter Beugungsordnung oder höher heraus gefiltert werden. Auch hier steht für jede Probe ein Filter zur Verfügung. Nach dem optischen Filter folgt die Halbleiterprobe im Strahlengang, an welche eine Spannung angelegt werden kann und somit der Stromfluss durch die Probe gemessen werden kann. Übersteigt die Photonennergie die Bandlückenenergie wird auf Grund der möglichen ABsorption ein Anstieg im Stromfluss erwartet. Hinter der Probe befindet sich ein Pyrodetektor , der das transmittierte Licht misst. Teil des Versuchaufbaus ist außerdem ein Lock-In-Verstärker.

\subsubsection{Durchführung}

Zunächst wird der Strahlengang überprüft und alle Geräte so justiert, dass der Strahlengang optimiert wird. Für jede Probe wird  (mit entsprechendem Filter und Gitter) das Winkelmaß zunächst genullt, und anschließend ein Winkelbereich von $-70\,^{\circ} - +70\,^{\circ} $ durchfahren während mit dem Programm "Logger-Pro" das Absorptions- und Transmissionsspektrum aufgenommen wird. Zusätzlich wird für jede Probe eine Untergrundmessung vorgenommen, in dem bei geschlossener Blende  Absorptions- und Transmissionsspektrum aufgenommen werden und die Strahlungsleistung der Lampe aufgenommen, indem die Probe aus dem Versuchsaufbau entfernt wird. Außerdem wird für jede Probe eine Fehlermessung durchgeführt, indem bei einer Winkeleinstellung über einen längeren Zeitraum gemessen wird. Für die erste Probe (Silizium) haben wir zwei Fehlermeswsungen durchgeführt, um zu untersuchen ob es sich um relative oder absolute Fehler handelt.
\subsection{Teil II}

\subsubsection{Versuchsaufbau}

\gra{TeilII_Aufbau}{Versuchsaufbau Teil II: Bewegung der Ladungsträgerwolke}\label{Aufbau2}

In Abbildung \ref{Aufbau2} ist der Versuchsaufbau zum Haynes-Shockley-Versuch zu sehen. Licht eines gepulstes Lasers wird hier durch ein Glasfaserkabel auf eine Germaniumprobe ($40\times 40\times 300\mathrm{mm}^3$, p-dotiert) geleitet.  Hierdurch ensteht eine Ladungsträgerwolke, welche durch die angelegte Spannung $U$ durch den Halbleiter wandert. Wir beobachten diese Wolke mit Hilfe eines Oszilloskops und einer Sonde.
Um ordentlich messen zu können, ist es erforderlich, dass die Frequenz des gepulsten Lasers und die des Oszilloskops synchronisiert sind, was über die Verbindung des Trigger-Eingangs des Oszilloskops mit dem Laser bezweckt wird.

Da die Probe wieder abkühlen muss, ist die Spannung $U$ nicht konstant. Viel mehr werden $98\%$ der Zeit einer Periode für das Abkühlen der Probe verwendet während der eigentliche Prozess in den restlichen $2\%$ der Zeit stattfindet.
Da die Treiberspannung deutlich größer ist als die Ladungsträgerspannung, muss diese mit einem \glqq shifted output\grqq\ herausgefiltert werden.

\subsubsection{Versuchsdurchführung}

In diesem Teil werden zwei verschiedene Messreihen aufgenommen. In einer ersten Messreihe wird der Abstand zwischen Lichtkabel und Nadel variiert, während die Spannung konstant bleibt. Es werden zehn Messungen für Abstände zwischen ca. $2-10$ mm durchgeführt. In einer zweiten Messreihe wird nun bei einem konstanten Abstand von ca. $4$ mm (inklusive Offset) die angelegte Spannung variiert. Hier werden elf Messungen für Spannungen zwischen $20$ V und $50$ V vorgenommen. Beide Messreihen sollten Messpunkte in möglichst regelmäßigen Abständen enthalten. 
\subsection{Teil III} 

\subsubsection{Aufbau}
\gra{TeilIII_Aufbau}{Versuchsaufbau Teil III: Halbleiterdetektoren}

Kernstück dieses Versuchsaufbaus ist der Halbleiterdetektor. Es stehen für diesen Versuchsteil zwei verscheidene Halbleiterdetektoren zur Verfügung:

\begin{itemize}
	\item Eine $n-n^+$-Silizium-Diode
	\item Ein CdTe-Kristall
\end{itemize} 

Es werden mit beiden Halbleiterdetektoren je die Spektren von  $^{57}$Co und $^{241}$Am aufgenommen. Durch Einstellen einer festen Messdauer wird eine einheitliche Normierung auf eine feste Messdauer überflüssig. \\
Trifft ein durch den Zerfall enstehendes $\gamma$-Quant auf den Halbleiter, so kann es seine Energie durch den Photoeffekt bzw. den Comptoneffekt ganz oder teilweise auf ein Elektron übertragen, welches so vom Valenzband in das Leiterband gehoben werden kann. Dieses kann durch Stöße seine Energie an weitere Elektronen übertragen. Die Anzahl der zusätzlichen Ladungsträger ist dabei proportional zur Ursprungsenergie des Photons.
Durch Anlegen einer Spannung kann so ein Strompuls gemessen werden. Dieser Strompuls wird vom Vorverstärker in einen Spannungssignal umgewandelt. Dieses Spannungssignal führt dann zum Aufladen eines Kondensator im Shaping Amplifier. Nachdem sich dieser wiederum über einen Widerstand entladen hat, wird das Signal über zwei RC-Filter gedämpft. Man erhält so einen scharfen Spannungspeak dessen Amplitude weiter proportional zur Ursprungsenergie des Photons ist. Der Multi Channel Analyzer ordnet damit verschiedene Energien den verschiedenen Channels zu und leitet dies an den Computer weiter mit dem wir nun schließlich Zerfallspektren aufzeichnen können.

\subsubsection{Durchführung}

Der Halbleiterdetektor wird in den vorgesehenen Kasten eingebaut und die radioaktive Probe daraufgestellt. Mit dem Computer-Programm ADMCA werden die Zerfalls-Spektren aufgenommen. Es werden vier Messungen durchgeführt mit einer Messzeit von jeweils einer Stunde:

\begin{itemize}
	\item Spektrum von $^{57}$Co mit Si-Detektor
	\item Spektrum von $^{57}$Co  CdTe-Detektor
	\item Spektrum von $^{241}$Am  mit Si-Detektor
	\item Spektrum von $^{241}$Am mit CdTe-Detektor
\end{itemize} 


\newpage
\section{Auswertung}
\subsection{Berechnen der Bandlückenenergien}
\graTwo[0.49]{bandluecke_ge_untergrund}{bandluecke_si_untergrund}{Untergrundmessung von a) Germanium und b) Silizium\label{untergrund}}
Zum Berechnen der Bandlückenenergien müssen wir zunächst den Untergrund von den gemessenen Spektren abziehen und diesen auf die Strahlungsleistung der Lampe normieren. Dafür benutzen wir folgende Rechnungen (\ref{srcbandluecke}):
\begin{align*}
	\text{Trans}_{\text{real}}&=\frac{\text{Trans}-\text{Untergrund}_{\text{Trans}}}{\text{Lampe}}\\
	\text{Abs}_{\text{real}}&=\frac{\text{Abs}-\text{Untergrund}_{\text{Abs}}}{\text{Lampe}}
\end{align*}
\graTwo[0.49]{bandluecke_ge_fehler}{bandluecke_si_fehler}{Messung zur Berechnung der Fehler von a) Germanium und b) Silizium\label{fehler}}
Als Untergrund verwenden wir den Mittelwert unserer gemessenen Untergrundwerte und den statistischen Fehler als Unsicherheit des Untergrundes (siehe Abbildung \ref{untergrund}). Aus unserer Fehlermessung (siehe Abbildung) erhalten wir die Unsicherheiten auf die gemessenen Spektren und auf das Lampenspektrum. Mit Gauß'scher Fehlerfortpflanzung erhalten wir nun die Fehler auf die realen Spektren (\ref{srcbandluecke}):
\begin{align*}
	s_{T,r}&=\text{Trans}_{\text{real}}\cdot\sqrt{\frac{s_T^2+s_{U,T}^2}{(\text{Trans}-\text{Untergrund}_{\text{Trans}})^2}+\frac{s_T^2}{\text{Lampe}^2}}\\
	s_{A,r}&=\text{Abs}_{\text{real}}\cdot\sqrt{\frac{s_A^2+s_{U,A}^2}{(\text{Abs}-\text{Untergrund}_{\text{Abs}})^2}+\frac{s_T^2}{\text{Lampe}^2}}
\end{align*}
Bei der Normierung auf die Strahlungsleistung stießen wir auf das Problem, dass die Datensätze der unterschiedlichen Spektren unterschiedliche x-Werte (Energie-Werte) hatten und wir die Spektren deshalb nicht direkt dividieren konnten. Also mussten wir das Spektrum der Messung der Lampenleistung auf die x-Werte der gemessenen Absorptions- und Transmissionsspektren umrechnen. Dies geschieht im Skript \code{lampe.R} (siehe \ref{srclampe}). Dazu suchen wir zu jedem x-Wert der Absorptions- und Transmissionsspektren den jeweils nächsten x-Wert des Lampenspektrums in positiver und in negativer x-Richtung und machen eine lineare Extrapolation aus den dazugehörigen y-Werten.

In den Abbildungen \ref{spektrumge} und \ref{spektrumsi} sieht man nun die bereinigten und normierten Spektren von Germanium und Silizium. Nun betrachten wir die negativen und positiven Bereiche rund um die Schnittpunkte der beiden Verläufe und berechnen die Schnittpunkte, um aus dem Betrag des x-Wertes die Bandlückenenergie zu erhalten (siehe Abbildungen \ref{speknahge} und \ref{speknahsi}).

Anschließend berechnen wir aus den errechneten Werten für die Bandlückenenergie aus dem positiven und dem negativen Bereich den gewichteten Mittelwert. Eine etwaige Verschiebung des Nullpunkts wird durch die Mittelung der beiden Werte eliminiert.

\paragraph{Bandlückenenergie von Germanium}
Die von uns berechnete Bandlückenenergie von Germanium beträgt: $E_{\text{Ge}}=0.68\pm0.05 \text{ eV}$
\paragraph{Bandlückenenergie von Silizium}
Die von uns berechnete Bandlückenenergie von Silizium beträgt: $E_{\text{Si}}=1.09\pm0.04 \text{ eV}$\\
\gra[0.7]{bandluecke_ge_spektrum}{Reales Spektrum von Germanium \label{spektrumge}}
\gra[0.7]{bandluecke_si_spektrum}{Reales Spektrum von Silizium \label{spektrumsi}}
\graTwo[0.49]{bandluecke_ge_negativ}{bandluecke_ge_positiv}{Nahaufnahmen des Germanium-Spektrums im a) negativen und b) positiven Bereich\label{speknahge}}
\graTwo[0.49]{bandluecke_si_negativ}{bandluecke_si_positiv}{Nahaufnahmen des Silizium-Spektrums im a) negativen und b) positiven Bereich\label{speknahsi}}

\subsection{Teil III: Halbleiterdetektor}

\subsubsection{Energiekalibrierung}

Anhand der aufgenommenen Spektren werden zunächst Energiekalibrierungen vorgenommen. Dafür werden in den aufgenommenen Spektren die Peaks der Zerfälle mit bekannten Energien ($59.5$ keV für $^{241}$Am und $122$ keV bzw. $136.5$ keV für $^{57}$Co) zunächst Gauß-gefittet (siehe ). Dies geschieht für den CdTe-Detektor und den Si-Detektor sowie die jeweiligen Proben getrennt. 
Beim Identifizieren der Peaks ist zu berücksichtigen, dass es durch verschiedene Effekte, die z.B. mit dem Compton-Effekt und dem Photoeffekt zu tun haben, besonders im Anfangsbereich zu einem starken Untergrund kommt. So führen Wechselwirkungen über den Compton-Effekt beispielsweise dazu, dass die gestreuten Photonen Einträge in einem weiten Bereich des Spektrums verursachen können und zwar bis zur sogenannten Compton-Kante, die der maximalen Energie eines gestreuten Photons entspricht (d.h. Streuwinkel von 180°). 

\gra{Am_CdTE_Teil3}{CdTe-Detektor: $59.5$ keV Peak von $^{241}$Am  }
\gra{Co_CdTe_Teil3}{CdTe-Detektor: $122$ keV bzw. $136.5$ keV von $^{57}$Co}
\gra{Am_Si_Teil3}{Si-Detektor: $59.5$ keV Peak von $^{241}$Am }
\gra{Co_Si_Teil3}{Si-Detektor: $122$ keV bzw. $136.5$ keV von $^{57}$Co}

\newpage

Wir erhalten für die verschiedenen Peaks mit ihren Gaußfits folgende Werte:



\begin{table}[h!]

\centering
\begin{tabular}{l|c|c}
&CdTe&Si\\\hline
Peak 1&

 $\mu    = 313.7\pm0.3$&$\mu    = 52.0\pm0.3$\\
&$ \sigma = 11.1\pm0.4$&$\sigma = 2.0\pm0.3$\\
&$ C = 46132\pm2381$&$C = 6244\pm1282$\\\hline

Peak 2&

 $\mu    = 647.5\pm0.3$&$\mu    = 620.9\pm0.3$\\
 &$\sigma = 10.6\pm0.4$&$\sigma = 11.6\pm0.6$\\
&$ C = 7252\pm373$&$C = 16283\pm9718$\\\hline

Peak 3&

 $\mu    = 723.8\pm1.0$&$\mu    = 695.4\pm1.5$\\
 &$\sigma = 9.4\pm1.3$&$\sigma = 13\pm2$\\
 &$C = 666\pm115$&$C = 13042\pm2798$\\\hline
\end{tabular}
\caption{Peaks der Zerfallspektren}
\end{table}

wobei wir Gaußfunktionen folgender Form verwendet haben:

\[c(x)=C\frac{1}{\sigma}exp\left[ -\frac{1}{2}\left(\frac{x-x_c}{\sigma}\right)^2\right] \]

Auf den Faktor $\frac{1}{\sqrt{2\pi}}$ haben wir hier verzichtet, da er für die Auswertung nicht relevant ist.

Anhand dieser führen wir nun die Energiekalibrierung durch, indem wir einen linearen Fit für die Energie-Channel-Paare tätigen.

\gra{CdTe_Kali_Teil3}{CdTe: Energiekalibrierung} \label{Linfit}
\gra{Si_Kali_Teil3}{Si: Energiekalibrierung} 

Aus den linearen Fits (siehe Abbildung \ref{Linfit}) ergeben sich für unsere Energiekalibrierung folgende Werte:

\begin{table}[h!] 
\centering
\begin{tabular}{l|c|c}

&CdTe&Si\\\hline
Energiekalibrierung&
 $b_{CdTe} = 5.330\pm0.010 channel/keV $& $b_{Si}    = 8.7\pm0.4 channel/keV $\\


 \end{tabular}
 \caption{Energiekalibrierung der Halbleiterdetektoren \label{b}}
\end{table}

\subsubsection{Absorptionsrate}

Wir wollen in einem nächsten Schritt nun die Absorptionsrate der Detektoren berechnen, die sich aus dem Verhätnis der Vorfaktoren $C$ des Gaußfits berechnet, wobei die Absorptionsflächen der Detektoren $a_{Si}=100mm^2$ $a_{CdTe}=23mm^2$ zu berücksichtigen sind:

\begin{align*}
r&=\frac{C_{Si}a_{CdTe}}{C_{CdTe}a{Si}}\\
\end{align*}

wobei sich der Fehler mit Gauß'scher Fehlerfortpfalnzung berechnet:

\begin{align*}
s_r&=r \sqrt{\left(\frac{s_{C_{Si}}}{C_{Si}}\right)^2+\left(\frac{s_{C_{CdTe}}}{C_{CdTe}}\right)^2}
\end{align*}

Wir erhalten für die verschiedenen Peaks folgende Ergebnisse:
\begin{table}[h!]

{\centering{}
\begin{tabular}{c||c|c}
					& Absorptionsrate r & Fehler	\\ \hline\hline
$59.5 eV$ Peak		& 0.031			    & 0.007		\\ \hline
$122.06 eV$ Peak	& 0.5 			    & 0.3	\\ \hline
$136.47 eV$ Peak	& 4.5 		     	& 1.2
\end{tabular}\\}
 \caption{Absorptionsrate
  der Halbleiterdetektoren}
\end{table}
\subsubsection{Energieauflösung}
In diesem Abschnitt soll nun noch die relative Energieauflösung berechnet werden, die sich aus dem Verhältnis der Halbwertsbreite eines Peaks zu dessen Lage berechnet. Für eine Gaußkurve berechnet sich die Halbwertszeit wie folgt:

\begin{align*}
FWHM = 2\sqrt{2\ln2}\sigma = 2.35 \sigma
\end{align*}

Damit gilt für die relative Energieauflösung:

\begin{align*}
RER(E) = \frac{FWHM_E}{E}=2.35\frac{\sigma_E}{E} = 2.35\frac{\sigma}{b \cdot E}
\end{align*}

wobei 
\begin{itemize}
\item $E =$ Energie der Photonen
\item $b =$ Energiekalibrierungsfaktor (siehe Tabelle \ref{b})
\item $\sigma_E =$ Standardabweichung in keV
\item $\sigma =$  Standardabweichung in Channel 
\end{itemize}

Der Fehler berechnet sich wie folgt:

\begin{align*}
s_{RER} = \sqrt{\left(\frac{s_b}{b}\right)^2+\left(\frac{s_\sigma}{\sigma}\right)^2}\cdot RER
\end{align*}

Als Ergebnis erhalten wir so für CdTe:

\begin{table}[h!]

{\centering{}
\begin{tabular}{c||c|c}
					& Energieauflösung & Fehler	\\ \hline\hline
$59.5 eV$ Peak		& 0.083		        & 0.003		\\ \hline
$122.06 eV$ Peak	& 0.0383		    & 0.0014	\\ \hline
$136.47 eV$ Peak	& 0.030	         	& 0.004
\end{tabular}\\}
 \caption{Energieauflösung CdTe}
\end{table}

und für Si:

\begin{table}[h!]

{\centering{}
\begin{tabular}{c||c|c}
					& Energieauflösung & Fehler	\\ \hline\hline
$59.5 eV$ Peak		& 0.0091		    & 0.0014 	\\ \hline
$122.06 eV$ Peak	& 0.0257			& 0.0017	\\ \hline
$136.47 eV$ Peak	& 0.026		     	& 0.004
\end{tabular}\\}
 \caption{Energieauflösung Si}
\end{table}

\newpage
\section{Zusammenfassung/Diskussion}




\subsection{Diskussion}\label{Diskussion}


\newpage
\section{Anhang}

\subsection{Sourcecode}
\subsubsection{bandluecke.R}\label{srcbandluecke}
\lstinputlisting[language=R]{r_files/bandluecke.R}
\subsubsection{lampe.R}\label{srclampe}
\lstinputlisting[language=R]{r_files/lampe.R}
\subsubsection{linearfit.R}\label{srclinearfit}
\lstinputlisting[language=R]{r_files/linearfit.R}
\subsubsection{konstift.R}\label{srckonstfit}
\lstinputlisting[language=R]{r_files/konstfit.R}

%\subsection{Einhorn}
%\gra[1]{Einhorn}{Ein Einhorn}



%\subsubsection{$\alpha$-Plateau Samarium}
%\lstinputlisting{data/Americium_1.TKA}


%\newpage
%\subsection{Quellcode (MATLAB)}
%\lstinputlisting[language=MATLAB]{Rohdaten/alpha.m}


%\begin{minipage}{\textwidth}
%\centering
%\includegraphics[width=0.9\textwidth]{figures/IMG_20151002_141014.jpg}
%\end{minipage}

\newpage
\listoffigures

%Literatur----------------------------------------------------------------------------------------------------------

%\cite{les}
\newpage
\thispagestyle{empty}
\begin{thebibliography}{9}

\bibitem{anleitungkhwz}
 (Zerfall von $^{57}$Co): http://hacol13.physik.uni-freiburg.de/fp/Versuche/FP1/FP1-6-KurzeHalbwertzeiten/Anleitung.pdf
  
\bibitem{staat}
  	Simon Amrein,
  	\emph{Staatsexamensarbeit: Halbleiter und Halbleiterdetektoren},\\
  	Albert-Ludwigs-Universität Freiburg,
  	Physikalisches Institut,
  	2008

\bibitem{chem}
	Caroline Röhr
	\emph{Skript zur Vorlesung: Methoden der anorganischen Chemie},\\
	Albert-Ludwigs-Universität Freiburg,
	Institut für anorganische und analytische Chemie,\\
	\emph{kein Herausgabedatum, da es sich um eine Website handelt (Stand: 22.08.15 17:16)},\\
	http://ruby.chemie.uni-freiburg.de/

\bibitem{anleitung}
  	S. Amrein (2008), K. Lohwasser und M. Köhli (2011), S. Kühn (2013)
  	\emph{Versuchsanleitung Fortgeschrittenen Praktikum I: Halbleiter},
  	Albert-Ludwigs-Universität Freiburg,
  	Physikalisches Institut
  
%\bibitem{molmasse}
%  \emph{http://www.convertunits.com/molarmass/<ELEMENTNAME AUF ENGLISCH>}, Stand 28.09.2015
  

\end{thebibliography}

\end{document}