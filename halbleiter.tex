\documentclass[12pt]{article}

\usepackage{fancyhdr}
\usepackage{geometry}
\usepackage{ucs}
\usepackage[utf8x]{inputenc}
\usepackage[T1]{fontenc}
\usepackage[ngerman]{babel}
\usepackage{amsmath,amssymb,amstext}
\usepackage{hyperref}
\usepackage{cancel}
\usepackage{dsfont}
\usepackage{physics}
\usepackage{lmodern}
\usepackage{enumerate}
\usepackage{enumitem}
\usepackage{graphicx}
\usepackage{listings, color}
\usepackage[labelfont=bf]{caption}
\usepackage{titling}


\lstset{basicstyle=\scriptsize,breaklines=true} %Quellcode mit Umlauten und ganz klein

\lstset{literate=
  {Ö}{{\"O}}1
  {Ä}{{\"A}}1
  {Ü}{{\"U}}1
  {ß}{{\ss}}2
  {ü}{{\"u}}1
  {ä}{{\"a}}1
  {ö}{{\"o}}1
}


%Geometrie----------------------------------------------------------------------------------------------------------

\geometry{a4paper, top=25mm, left=15mm, right=15mm, bottom=25mm,headsep=10mm, footskip=10mm}
\pagestyle{fancy}
\setlength{\parindent}{0pt} %Zeileneinrückung

\fancyhf{} %Setzt voreingestellte Kopf-und Fußzeilen-Eigenschaften zurück

\lhead{\nouppercase{\leftmark}}
\chead{}
\rhead{\thepage}

\lfoot{}
\cfoot{}
\rfoot{}

\title{\vspace{0cm}{\Huge Fortgeschrittenen-Praktikum I:\\ \vspace{1cm}  Halbleiter}}
\author{Saskia Bondza\\Simon Stephan}
\date{Durchgeführt am 07.09.2016}

\pretitle{%
  \begin{center}
  \LARGE
  \includegraphics[width=6cm,]{figures/siegel}\\[\bigskipamount]
}
\posttitle{\end{center}}

%neue Commands----------------------------------------------------------------------------------------------------------
\newcommand{\nab}{\vec{\nabla}} %direkter Befehl mit Vektorpfeil
\newcommand{\gra}[3][0.7]{
	\begin{minipage}[h!]{\textwidth}
		\centering
		\includegraphics[width=#1\textwidth]{figures/#2.png}
		\captionof{figure}{#3}
	\end{minipage}
	\vskip 30 pt
	}


%Titel,Inhalt----------------------------------------------------------------------------------------------------------

\begin{document}
\pagenumbering{gobble} %verstecke Seitenzahl
\maketitle
\newpage

\thispagestyle{empty}
\tableofcontents
\newpage

%Schreiben----------------------------------------------------------------------------------------------------------
\pagenumbering{arabic} %verstecke Seitenzahl
\section{Einleitung}

In diesem Experiment soll in drei verschiedenen Versuchsteilen die Funktionsweise von Halbleitern und Halbleiterdetektoren untersucht werden. Silizium wird hier als Beispiel für einen reinen Halbleiter untersucht und Kadmiumtellurid als Beispiel für einen Verbindungshalbleiter.
In \textbf{Teil I} des Versuchs wird die Bandlückenenergie, wichtiges Charakteristikum von Halbleitern, gemessen, die die Auflösung eines Halbleiterdetektors bestimmt.\\
In \textbf{Teil II} des Versuchs geht es um die Beobachtung der Ausbreitung
von Ladungsträgern im Halbleiter mit dem Haynes-Shockley-Experiment und in \textbf{Teil III} des Experiments werden mit Halbleiterdetektoren radioaktive Spektren aufgenommen und untersucht, wie gut sich Halbleiterdetektoren hierfür eignen.


\newpage
\section[Theoretische Grundlagen]{Theoretische Grundlagen}

Dieser Teil orientiert sich an [].

\subsection{Interferenz von Photonen, optisches Gitter}

Trifft eine ebene Welle auf einen Spalt, so ist nach dem Huygene'schen Prinzip jeder Punkt der Wellenfront Ausgangspunkt einer Elementarwelle, wobei sich die neue Wellenfront durch Überlagerung sämtlicher Elementarwellen ergibt. Trifft eine ebene Welle auf mehrere Spalte, ein optisches Gitter, beobachtet man hinter dem Gitter Interferenz, ein charakteristisches Wellenphänomen.  Ist der Gangunterschied zweier Wellen ein ganzzahliges Vielfaches ihrer Wellenlänge kommt es zur konstruktiven Interferenz, ist der Gangunterschied ein ungerades, halbzahliges Vielfaches der Wellenlänge kommt es zur destruktiven Interferenz. Es gilt also folgender Zusammenhang für die konstruktive Interferenz am Gitter:

\begin{align*}
& n\cdot\lambda=d\cdot\sin(\theta)
\end{align*}
Hierbei ist $n$ die Beugungsordnung, $\theta$ der Einfallswinkel, $\lambda$ die Wellenlänge und $d$ der Spaltabstand.
Man unterscheidet zwei Arten von optischen Gittern: Transmissionsgitter und Reflexionsgitter; Transmissionsgitter bestehen aus parallelen Spalten, Reflexionsgitter aus reflektierenden Graten. Hier handelt es sich um ein Reflexionsgiterr. Wir betrachten nun nur die erste Beugungsordnung. Für verschiedene Anstellwinkel des optischen Gitters ergibt sich dann die Energien der Photonen zu:

\[E(\varphi)=\frac{hc}{2d\sin(\varphi)cos(\frac{\psi}{2})}\]

In 4.1 wird genauer erklärt wie sich ein optisches Gitter so als Spektrometer verwenden lässt.
\subsection{Das Bändermodell}

Bei einem einzelnen Atome sind die Energieniveaus der Elektronen diskret. Nähert man nun zwei Atome so nah an, dass sich ihre Orbitale überlappen, müssen sich die Energieniveaus gemäß dem Pauliprinzip aufspalten. In einem Kristall wechseln nun der Art viele Atome miteinander, dass auf diese Art und Weise Energiebänder enstehen (s. Abbildung ). Das energetisch höchste Band, das im Grundzustand besetztist heißt Valenzband, das erste unbesetzte Band heißt Leitungsband. Der Energieabstand zwischen diesen beiden Bändern wird als Bandlücke bezeichnet. Bei $T = 0$K ist das Valenzband voll besetzt während im Leitungsband keine Ladungsträger sind. Durch Anlegen einer Spannung oder Absorption von Photonen können Elektronen jedoch in das Leiterband gelangen. 
Die charakteristische Größe der Bandlückenenergie definiert ob ein Stoff ein Isolator, Halbleiter oder Leiter ist. Isolatoren haben eine besonders große Bandlüclenenergie, die mehrere $eV$ beträgt. Halbleiter haben typischerweise eine Bandlücke von weniger als einem $eV$, sie isolieren nur bei sehr niedrigen Temperaturen. Bereits thermische Anregungen reichen bei höheren Temperaturen für die Elektronen aus, um ins Leitungsband gehoben zu werden, Zurück bleibt bei diesem Vorgang ein positiv geladenes Loch, man spricht von einem Elektron-Loch-Paar. Bei Leitern überlappen sich beide Energiebänder, sodass stets genug Ladungsträger im Leitungsband sind, um einen Stromfluss zu gewährleisten.

\subsection{Bewegung von Ladungsträgern im Halbleiter}

\subsection{Direkte und indirekte Halbleiter}

Untersucht man die Energien der Bänder in Abhängikeit des Impulses ergeben sich komplizierte Verläufe, aus denen sich eine natürliche Unterscheidung von Halbleitern ableiten lässt: Liegt das Maximum des Valenzbandes beim gleichen Impuls wie das Minimum des Leitungsbandes spricht man von direkten Halbleitern, liegen Maximum und Minimum bei verschiedenen Impulsen, spricht man von indirekten Halbleitern.
Bei direkten Halbleitern können Elektronen einfach durch die Aufnahme der Bandlückenenergie (Übertrag durch ein Photon von ausreichender Energie $E_{Photon}=\hbar\omega\geq E_g$) ins Leitungsband gehoben werden während bei indirekten Halbleitern durch die Verschiebung der Extrema um $\Delta p$ ein Elektron zusätzlich zur Energieaufnahme auch noch seinen Impuls ändern muss. Dies geschieht durch Erzeugung bzw. Vernichtung von Phononen, d.h. Gitterschwingungen.

\subsection{Extrinsische Halbleiter und Dotierung}

In der Theorie wird meist von intrinsischen Halbleitern ausgegangen, also perfekten Kristallen ausgegangen.  Diese lassen sich allerdings technisch nicht erzeugen. Kristalldefekte, z.B.Beschädigte Elementarzellen, Verschiebung ganzer Kristallebenen oder die Verunreinigung durch Fremdatom beeinflussen die Kristallstruktur und können zur Bildung von Energieniveaus innerhalb der Bandlücken führen, was einen nicht zu vernachlässigenden Effekt auf die Elektronen-Loch-Rekombination hat, und somit die Leitfähigkeit beeinflusst.\\
Normalerweise geht ein Halbleiteratom Bindungen mit vier Nachbaratomen ein. Bei der Verunreinigung mit Fremdatomen z.B. Phosphor mit fünf Valenzelektronen oder Aluminium mit drei Valenzelektronen steht ein quasi-freier Ladungsträger zur Verfügung: Bei Phosphor ist dies das fünfte, schwachgebundene Elektron, bei Aluminium steht ein zusätzliches Loch als Ladungsträger zur Verfügung. Bei Fremdatome, die ein zusätzliches Elektron beisteuern, wie in diesem Beispiel Phosphor, spricht man von Donatoren. Bei solchen, die ein zusätzliches Loch beisteuern, wie hier Aluminium, spricht man von Akzeptoren.
Oft wird diese Verunreinigung absichtlich vorgenommen (typischerweise ca. 1 Fremdatom auf $10^6$Atome), da sie sich positiv auf die Leitfähigkeit des Halbleiters auswirken kann. Diesen Vorgang nennt man Dotierung. Donatoren-Halbleiter nennt man n-Typ, Akzeptor-Halbleiter heißen p-Typ.
\subsection{Halbleiter-Diode}

\subsection{Wechselwirkung von Strahlung mit Materie}

Die Wechselwirkung von Strahlung mit Materie wird im Wesentlichen durch drei verschiedene Effekte beschrieben: \textbf{Photoeffekt}, \textbf{Comptoneffekt} und \textbf{Paarbildung}.
Der Effekt der Paarbildung tritt für Energien über $1,022$ MeV auf, die in diesen Versuchen nicht erreicht werden, sodass wir diesen Effekt hier nicht weier erläutern.

\paragraph{Photoeffekt}

\paragraph{Comptoneffekt}
\subsection{Theorie zum Haynes-Shockley Experiment}



 
\newpage
\section[Versuchsaufbau- und Durchführung]{Versuchsaufbau- und Durchführung} 








 





\newpage
\section{Auswertung}



\newpage
\section{Zusammenfassung/Diskussion}




\subsection{Diskussion}\label{Diskussion}


\newpage
\section{Anhang}






%\subsubsection{$\alpha$-Plateau Samarium}
%\lstinputlisting{data/Americium_1.TKA}


%\newpage
%\subsection{Quellcode (MATLAB)}
%\lstinputlisting[language=MATLAB]{Rohdaten/alpha.m}


%\begin{minipage}{\textwidth}
%\centering
%\includegraphics[width=0.9\textwidth]{figures/IMG_20151002_141014.jpg}
%\end{minipage}

\newpage
\listoffigures

%Literatur----------------------------------------------------------------------------------------------------------

%\cite{les}
\newpage
\thispagestyle{empty}
\begin{thebibliography}{9}

\bibitem{anleitung}
 (Zerfall von $^{57}$Co): http://hacol13.physik.uni-freiburg.de/fp/Versuche/FP1/FP1-6-KurzeHalbwertzeiten/Anleitung.pdf
  
\bibitem{physchem}
	Gerd Weldler, Hans-Joachim Freund,
	\emph{Lehrbuch der Physkalischen Chemie, Band 1},
	6th edition,
	John Wiley \& Sons,
	2012

\bibitem{nuclear}
Bernard L. Cohen,
\emph{Concepts of Nuclear Physics},
 New York usw.: McGraw-Hill 1971
  
%\bibitem{molmasse}
%  \emph{http://www.convertunits.com/molarmass/<ELEMENTNAME AUF ENGLISCH>}, Stand 28.09.2015
  

\end{thebibliography}

\end{document}